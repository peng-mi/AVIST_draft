\firstsection{Introduction}

\maketitle


As the rapid growing of data, interactive visual analysis becomes a key component to gain usable insights from massive data. To support sensemaking of big data, exploratory visual analysis emphasizes three aspects. Firstly, \textsl{flexible data filtering} (e.g., cross-filtering~\cite{ weaver2008, weaver2010}) empower analysts to ``prune" large datasets to gain buried relationships. Secondly, \textsl{fine data details} should be provided on demand, so that analysts can capture certain subtle abnormal activities. Lastly, \textsl{exploration data analysis} is usually an interactive and iterative process. Thus high performance is necessary for exploration of big data, which requires real-time frame-rates. For example, when cyber security analysts attempt to identify attacks from millions of records in network traffic logs, they have to filter data records by iteratively using different data attributes to finally narrow down to the suspicious records.  

However, there are many challenges to achieve these goals for analyzing big data.
 When the data becomes bigger, it cannot feed into memory or the computation speed cannot afford real-time performance. In order to feed more data into memory, Tableau~\cite{tab} highlights the Tableau Data Engine (TDE)~\cite{Wesley}. The TDE is a specialized column-oriented format, which has high data compression ratio (considering the repeating values in the columns) and can provide high level data aggregation information. To achieve fast visual querying of big data, imMens~\cite{2013-immens} utilizes the GPU parallel computing to improve performance. Firstly, it aggregates data into data tiles by the data cube technique. Secondly, it uses WebGL for data processing and rendering on the GPU. However, both Tableau and imMens only consider visual exploring the high level data aggregation information of big data. To flexibly investigate fine-grained data details, such as identifying buried data relationships by complexity filtering, those methods cannot directly be applied, and new techniques need to be investigated.   

This paper tackles the problem of  visual exploration of big data,  and aims to help analysts  get the fine-grained details by exploratory visual analysis of big datasets. To achieve this, we propose an animated visualization tool - AVIST, which runs on a  commodity desktop computer for analyzing millions of multi-dimensional data records per second by leveraging the GPU resources. We emphasize ``\textit{animation}" in the context of time-series multidimensional datasets for identifying data temporal behaviors. The performance of AVIST is judged by FPS (frame per second), which emphasizes the high velocity feature. In summary, our key contributions in this paper are as follows:

1) We propose a GPU-centric design for interactive visual exploration of big datasets, which emphasizes that data storage and computation are done by the GPU.

2) We design a data dependency graph for characterizing data transformations on the GPU, which supports data aggregation and visualization on demand.

3) We implement  AVIST following the GPU-centric design as a proof of concept. AVIST emphasizes  the  animation and cross-filtering interactions for slicing big data into small data, and the GPU parallel computing for transforming raw data into visual primitives. 

4) We present two usage scenarios to demonstrate that  AVIST can help analysts identifying abnormal behaviors of large network flow and inferring new hypotheses for international trade transactions.     


